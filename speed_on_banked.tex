\documentclass{article}
\usepackage{amsmath}
\usepackage{tikz}
\usepackage[margin=1in]{geometry}


\title{Cars 4 -- Lightnight McQuation}
\author{Rhitt C}
\date{Feb 2026}


\begin{document}
\maketitle

\section*{Abstract}
Lightning McQueen, old and rusty, decides to officially retire from the racing scene, leaving Cruz to carry on his legacy.

In search of a new career, he thinks back to all the crazy stunts he'd pulled over the years and realises he has no idea how they were even possible physically.
So he has a crazy idea: why not become a physicist and research stunt mechanics?

Who knows, maybe a strong understanding will lay the foundations for a future NASCAR training business $\dots$

\vspace{9 mm}

Note to self: learn simulink and experiment further with simulations

\section{The Basics of Friction -- Walking}

\section{First Paradigm -- Skidding and Drifting}

\section{Second Paradigm -- Banked Tracks}

\subsection{Deriving a general formula for Friction and Normal forces}
The first constraint is that the car travels in a purely horizontal plane 
  i.e. there is no vertical movement and so net vertical force on the car is $\vec{0}$.

For ease of calculation, we deal with both friction down the slope and up the slope together using $(\pm)$ for its components, taking care to use $(\mp)$ where appropriate to maintain consistency of signs.

Setting the upwards direction as positive, we have the following:
\begin{align*}
    \sum \vec{F}_y &= \vec{0} \\
    \vec{0}&=\vec{F_g} + \vec{F}_{fy} + \vec{F}_{Ny} \\
    \vec{0}&=-mg \mp |F_f|\sin\theta + |F_N|\cos\theta \tag{E1}\\
    (\text{E1})\cdot \cos\theta\implies\qquad\qquad 
        \vec{0} &=-mg\cos\theta \mp |F_f|\sin\theta\cos\theta + |F_N|\cos^2\theta \tag{E2}\\
\end{align*}


The second constraint is that the car undergoes uniform circular motion.

Setting the direction in towards the center of the velodrome to be positive, we must have positive centripetal force and hence:
\begin{align*}
    \sum \vec{F}_x &= +\; m\frac{v^2}{r} \\
    \vec{F}_{fx} + \vec{F}_{Nx}&=m\frac{v^2}{r} \\
    \vec{0}&= m\frac{v^2}{r} - \vec{F}_{fx} - \vec{F}_{Nx} \\
    \vec{0}&= m\frac{v^2}{r} \mp |F_f|\cos\theta - |F_N|\sin\theta \tag{E3}\\
    (\text{E3})\cdot \sin\theta\implies\qquad\qquad 
        \vec{0}&= m\frac{v^2}{r}\sin\theta \mp |F_f|\sin\theta\cos\theta - |F_N|\sin^2\theta \tag{E4}\\
\end{align*}


Now we combine to obtain an equation involving $F_f$ but not $F_N$

\begin{align*}
    \text{(E4)} - \text{(E2)} \implies \qquad
        \vec{0}-\vec{0}  &=  m\frac{v^2}{r}\sin\theta + mg\cos\theta + (\vec{0}) + |F_N|\cdot (-\sin^2\theta - \cos^2\theta) \\ 
        \vec{0} &= m\frac{v^2}{r}\sin\theta + mg\cos\theta + |F_N|\cdot(-1) \\ 
        |F_N| &= m \left(\frac{v^2}{r} \sin\theta + g\cos\theta \right) \tag{E5}
\end{align*}


This gives us the normal force! Now, plugging into (E1), we find the friction force

\begin{align*}
    \text{(E5)}\rightarrow\text{(E1)}\implies\quad\quad
        \vec{0}&=-mg \mp |F_f|\sin\theta + \left(mg\cos\theta + m\frac{v^2}{r}\sin\theta\right)\cos\theta \\
        \pm |F_f|\sin\theta  &= m\frac{v^2}{r}\sin\theta\cos\theta + mg (-1+\cos^2 \theta) \\
                             &= m\frac{v^2}{r}\sin\theta\cos\theta + mg (-\sin^2\theta) \\
        |F_f| &= \pm m\left(\frac{v^2}{r} \cos\theta - g\sin\theta\right)
\end{align*}


Summarising our results thus far, we have,

\begin{center}
\boxed{|F_f| = \pm m\left( \frac{v^2}{r} \cos\theta - g\sin\theta\right)\qquad\text{and}\qquad |F_N| = m\left(\frac{v^2}{r}\sin\theta + g\cos\theta\right)}

\end{center}

where the ($+$) for the ($\pm$) in $|F_f|$ is when directed down the slope and the ($-$) is when up the slope.

\subsection{Deriving Range of Speeds}

We now wish to find the maximum value of $|v|$ that meets our constraints for horizontal circular motion. This occurs when $F_f$ is at its maximum down the slope (i.e. ($\pm$) replaced with ($+$)), as this means $F_f$ and $F_N$ both have maximal centre-seeking components and thus contribute as much as possible to the centripetal force. Hence, we have:

\begin{align*}
|F_f|_{(+)} &\le \mu |F_N|\\
m\left(\frac{v^2}{r}\cos\theta - g \sin\theta\right)
&\le \mu m \left(\frac{v^2}{r}\sin\theta + g \cos\theta\right)\\
\frac{v^2}{r}\cos\theta - g \sin\theta
&\le \mu  \left(\frac{v^2}{r}\sin\theta + g \cos\theta\right)\\
\frac{v^2}{r}\left(\cos\theta - \mu\sin\theta\right)
&\le g (\sin\theta + \mu\cos\theta)\\
\frac{v^2}{r}&\le g\frac{\sin\theta + \mu\cos\theta}{\cos\theta - \mu\sin\theta}\\
|v|&\le \sqrt{rg\left(\frac{\sin\theta + \mu\cos\theta}{\cos\theta - \mu\sin\theta}\right)}
\end{align*}

Similarly, to minimise $|v|$, we want $F_f$ to be maximally directed up the slope (i.e. replace ($\pm$) with ($-$)). This minimises $F_N$ and also leads to a maximal component of $F_f$ that is centre-opposing, thereby minimising centripetal force.

Thus,

\begin{align*}
|F_f|_{(-)} &\le -\mu |F_N|\\
-m\left(\frac{v^2}{r}\cos\theta - g \sin\theta\right)
    &\le \mu m\left(\frac{v^2}{r}\sin\theta + g \cos\theta\right)\\
\frac{v^2}{r} \left(\cos\theta + \mu\sin\theta\right)
&\ge  g (\sin\theta - \mu\cos\theta)\\
|v| &\ge
\sqrt{rg\left(\frac{\sin\theta - \mu\cos\theta}{\cos\theta + \mu\sin\theta}\right)}
\end{align*}

Therefore the full range of speeds for which static friction and the normal force can ensure horizontal circular motion is summarised as

\begin{center}\boxed{\sqrt{rg\left(\frac{\sin\theta + \mu\cos\theta}{\cos\theta - \mu\sin\theta}\right)} \le
|v| \le
\sqrt{rg\left(\frac{\sin\theta - \mu\cos\theta}{\cos\theta + \mu\sin\theta}\right)}
}\end{center}


or, dividing out $\cos\theta$ to involve $\tan\theta$ only:
\begin{center}\boxed{\sqrt{rg\left(\frac{\tan\theta + \mu}{1 - \mu\tan\theta}\right)} \le
|v| \le
\sqrt{rg\left(\frac{\tan\theta - \mu}{1 + \mu\tan\theta}\right)}
}\end{center}


\section{New Business -- Lightnight McQuation}


\end{document}